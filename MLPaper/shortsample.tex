% This is the "preamble" of the document. This is where the format options get set.
% Pro-tip: things following the % mark will not be compiled by LaTeX. I'll be using them extensively to explain things as we go.
% Note: not to scare you off of LaTeX, but it's normal to have problems. And ya girl has been having some. I've included the copyright info at the bottom of the document from the guy who wrote this package, because his documentation doesn't entirely match how it's actually used. So this is a combination of his working preamble along with my added commentary or explanation. 
%% 


\documentclass[stu,12pt,floatsintext]{apa7}
% Document class input explanation ________________
% LaTeX files need to start with the document class, so it knows what it's using
% - This file is using the apa7 document class, as it has a lot of the formatting built in
% There are two sets of brackets in LaTeX, for each command (the things that start with the slash \ )
% - The squiggle brackets {} are mandatory for executing the command
% - The square brackets [] are options for that command. There can be more than one set of square brackets for some commands
% Options used in this document (general note - for each of these, if you want to use the other options, swap it out in that spot in the square brackets):
% - stu: this sets the `document mode' as the "student paper" version. Other options are jou (journal), man (manuscript, for journal submission), and doc (a plain document)
% --- The student setting includes things like 'duedate', 'course', and 'professor' on the title page. If these aren't wanted/needed, use the 'man' setting. It also defaults to including the tables and figures at the end of the document. This can be changed by including the 'floatsintext' option, as I have for you. If the instructor wants those at the end, remove that from the square brackets.
% --- The manuscript setting is roughly what you would use to submit to a journal, so uses 'date' instead of 'duedate', and doesn't include the 'course' or 'professor' info. As with 'stu', it defaults to putting the tables and figures at the end rather than in text. The same option will bump those images in text.
% --- Journal ('jou') outputs something similar to a common journal format - double columned text and figurs in place. This can be fun, especially if you are sumbitting this as a writing sample in applications.
% --- Document ('doc') outputs single columned, single spaced text with figures in place. Another option for producing a more polished looking document as a writing sample.
% - 12pt: sets the font size to 12pt. Other options are 10pt or 11pt
% - floatsintext: makes it so tables and figures will appear in text rather than at the end. Unforunately, not having this option set breaks the whole document, and I haven't been able to figure out why. IT's GREAT WHEN THINGS WORK LIKE THEY'RE SUPPOSED TO.

\usepackage[american]{babel}

\usepackage{csquotes} % One of the things you learn about LaTeX is at some level, it's like magic. The references weren't printing as they should without this line, and the guy who wrote the package included it, so here it is. Because LaTeX reasons.
\usepackage[style=apa,sortcites=true,sorting=nyt,backend=biber]{biblatex}
% biblatex: loads the package that will handle the bibliographic info. Other option is natbib, which allows for more customization
% - style=apa: sets the reference format to use apa (albeit the 6th edition)
\DeclareLanguageMapping{american}{american-apa} % Gotta make sure we're patriotic up in here. Seriously, though, there can be local variants to how citations are handled, this sets it to the American idiosyncrasies 
\addbibresource{bibliography.bib} % This is the companion file to the main one you're writing. It contains all of the bibliographic info for your references. It's a little bit of a pain to get used to, but once you do, it's the best. Especially if you recycle references between papers. You only have to get the pieces in the holes once.`

\usepackage[T1]{fontenc} 
\usepackage{mathptmx} % This is the Times New Roman font, which was the norm back in my day. If you'd like to use a different font, the options are laid out here: https://www.overleaf.com/learn/latex/Font_typefaces
% Alternately, you can comment out or delete these two commands and just use the Overleaf default font. So many choices!


% Title page stuff _____________________
\title{Machine Learning Algorithms and ACL Injury: A Sporting Minority Report?} % The big, long version of the title for the title page
\shorttitle{Machine Learning and ACL Injury} % The short title for the header
\author{Rowan Olafson}
\duedate{April 20, 2024}
% \date{January 17, 2024} The student version doesn't use the \date command, for whatever reason
\affiliation{Univresity of Calgary}
\course{KNES 381} % LaTeX gets annoyed (i.e., throws a grumble-error) if this is blank, so I put something here. However, if your instructor will mark you off for this being on the title page, you can leave this entry blank (delete the PSY 4321, but leave the command), and just make peace with the error that will happen. It won't break the document.
\professor{Dr. Holash}  % Same situation as for the course info. Some instructors want this, some absolutely don't and will take off points. So do what you gotta.

%\keywords{APA style, demonstration} % If you need to have keywords for your paper, delete the % at the start of this line

\begin{document}
\maketitle % This tells LaTeX to make the title page

% \section{Introduction} This command is commented out, because I was taught it was redundant to have the paper's title and introduction together. If your instructor wants it to say "Introduction", delete the % at the start

While Machine Learning has been around nearly since the advent of digital computers, in the last 20 years these specialized algorithms have been seeing widespread use \parencite{MLhistory}. Researchers in Kinesiology have begun using this technology to tackle one of the biggest issues in sporting - ACL injury. ACL injuries have varied outcomes, but almost always result in the loss of playing time and long term consequences for the athlete, both financially and in sporting opportunities. ACL injuries carry a heavy financial cost to the health care system, can result in major losses of money for both athletes and sporting clubs, and reduce overall quality of life in those affected \parencite{Filbay_2022}. Primary prevention is key in nullifying the effects of ACL injury, but how can we prevent these from happening? Using ML, research have been searching for a method of accurately predicting future ACL injury risk to allow for more targeted and specific prevention strategies. To gain a better understanding of this topic and its future use, we will discuss what ML actually is, how this technology is being applied to the problem of ACL injuries, and how this approach lends itself to specific advantages or drawbacks.

So, what exactly is ML? The central idea of ML is to emulate how biological organisms process sensory information, and use it to adapt their responses in a changing environment \parencite{MLhistory, WhatisML}. This flexibility of process is absent in traditionally coded programs, which are designed to produce a specific outcome from specific input.   ML is a computational process by which an algorithm uses input data to generate a desired outcome or output \parencite{WhatisML}. These programs are not hard coded to produce these outcomes, rather, they are made to "learn" through repetition and feedback, and alter their architecture to adapt to the programmers desired outcome \parencite{WhatisML}. By weighting the output of these algorithms based on how close they are to the desired response, programmers can use a “reward system” to inform the algorithm on how they want it to change its responses. This restructuring or adaptation allows ML programs to not only produce optimal outcomes for the dataset it was trained with, but also allows it to generalize and produce outcomes with novel datasets.  There are 2 main upsides to this method of data analysis; these programs can sub-in for humans and work with large datasets which would be otherwise be complex and extremely labourius, and ML programs have the ability to detect more subtle and complex patterns in the data, which humans may otherwise miss \parencite{WhatisML}. 

When applying ML to ACL injury, researchers have approached this issue in two main ways. The first approach is using ML algorithms to analyze datasets, with several different parameters that are associated with ACL injury and develop risks profile for athletes who either have or do not have these risk factors \parencite{Jauhiainen_2022,Taborri_2021}. These types of studies then recommend the implementation of specific training programs to reduce the risk of ACL injury in athletes assessed to be at risk. By designing training programs with direction from the algorithm, researchers can proactively treat the risk factors for identified for each individual \parencite{Taborri_2021}. The second approach is using ML to validate existing risk factors associated with ACL injury, or identify new ones \parencite{Jauhiainen_2020, Kokkotis_2022}. By utilizing several different prediction models, researchers were able to find a set of consistent injury predictors. Importantly, while the actual predicting power of the models may vary, consistent appearance of specific factors lends creedence to their association with injury \parencite{Jauhiainen_2020,Kokkotis_2022}. This second approach allows for the improvement of above the prediction models of the first approach. By feeding these algorithms with more specific risk factors, their predictions will likely improve \parencite{Jauhiainen_2020}.

A Major issue with this line of research, is the type of data used to inform the prediction models. While these prediction models can be highly accurate with the data they are trained/ presented with, the quality of that data will effect the outcome of the prediction. The data these algorithms have been trained on have been called into question. As outlined by Jauhiainen and collaborators (2022), previous prediction studies had relied on biomechanical assessment data in conjunction with anthropometric and strength data to make predictions of injury risk. For example, the authors note that knee abduction movements have been used as a proxy for knee injury risk as movements with high amounts of abduction have been associated with ACL injury in the past. While risk factors like knee abduction have been identified in previous studies, these studies have often been extremely small and only provide a statistical association with injury, and do not necessarily provide strong predictive powers\parencite{Jauhiainen_2022}. Additionally, the external validity of ML algorithm predictions has been called into question \parencite{Martin_2022}. These programs are not universally usable for all athletes in all parts of the globe. While some studies have demonstrated external validity from there original participants, authors are careful to generalize their results completely, and often restrict their claim of validity to a specific region \parencite{Martin_2022}.

ML is an extremely useful tool, however, it needs to be applied properly in order to achieve accurate results. These methods have shown promise in helping further the field of sports medicine, and may eventually help to reduce injury burden by allowing practitioners and coaches to proactively treat injury risk factors. With where we currently stand in the use of ML, it is important we stay mindful of what type of data we train these algorithms with, and avoid spreading claims of injury prediction beyond specific groups. With further refinement of this technology and the way we use it, we can surpass these obstacles. At the moment, ML use in the space of injury prediction and prevention requires more research, yet it is a promising new tool that may help revolutionize the way we think about injuries in sport.

\end{document}

%% 
%% Copyright (C) 2019 by Daniel A. Weiss <daniel.weiss.led at gmail.com>
%% 
%% This work may be distributed and/or modified under the
%% conditions of the LaTeX Project Public License (LPPL), either
%% version 1.3c of this license or (at your option) any later
%% version.  The latest version of this license is in the file:
%% 
%% http://www.latex-project.org/lppl.txt
%% 
%% Users may freely modify these files without permission, as long as the
%% copyright line and this statement are maintained intact.
%% 
%% This work is not endorsed by, affiliated with, or probably even known
%% by, the American Psychological Association.
%% 
%% This work is "maintained" (as per LPPL maintenance status) by
%% Daniel A. Weiss.
%% 
%% This work consists of the file  apa7.dtx
%% and the derived files           apa7.ins,
%%                                 apa7.cls,
%%                                 apa7.pdf,
%%                                 README,
%%                                 APA7american.txt,
%%                                 APA7british.txt,
%%                                 APA7dutch.txt,
%%                                 APA7english.txt,
%%                                 APA7german.txt,
%%                                 APA7ngerman.txt,
%%                                 APA7greek.txt,
%%                                 APA7czech.txt,
%%                                 APA7turkish.txt,
%%                                 APA7endfloat.cfg,
%%                                 Figure1.pdf,
%%                                 shortsample.tex,
%%                                 longsample.tex, and
%%                                 bibliography.bib.
%% 
%%
%%
%% This is file `./samples/shortsample.tex',
%% generated with the docstrip utility.
%%
%% The original source files were:
%%
%% apa7.dtx  (with options: `shortsample')
%% ----------------------------------------------------------------------
%% 
%% apa7 - A LaTeX class for formatting documents in compliance with the
%% American Psychological Association's Publication Manual, 7th edition
%% 
%% Copyright (C) 2019 by Daniel A. Weiss <daniel.weiss.led at gmail.com>
%% 
%% This work may be distributed and/or modified under the
%% conditions of the LaTeX Project Public License (LPPL), either
%% version 1.3c of this license or (at your option) any later
%% version.  The latest version of this license is in the file:
%% 
%% http://www.latex-project.org/lppl.txt
%% 
%% Users may freely modify these files without permission, as long as the
%% copyright line and this statement are maintained intact.
%% 
%% This work is not endorsed by, affiliated with, or probably even known
%% by, the American Psychological Association.
%% 
%% ----------------------------------------------------------------------